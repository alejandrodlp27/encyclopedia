%%% EXERCISE 10
\HRule
\subsection*{Exercise 1}
\subsection*{%
Let $M$ be an $R$-module. Prove that the following are equivalent:
\begin{enumerate}
	\item $M$ is noetherian (ie. every submodule is finitely generated).
	\item Every ascending chain $M_1\subseteq M_2\subseteq\cdots$ of submodules of $M$ terminates.
	\item Every set of submodules of $M$ has a maximal element with respect to inclusion.
	\item Given any sequence $\{f_n\}$ of elements of $M$ there exists an $m\in\NN$ such that for all $n>m$
		$f_n$ is a linear combination of $\{f_1,\ldots,f_m\}$. That is
		\[
			f_n=\sum_{i=1}^{m}a_i f_i \quad (a_i\in R).
		\]
\end{enumerate}
} 

%%% ANSWER
\begin{proof} $\;$\\
\begin{enumerate}
	\item[$1\then 2$] Let $M_1\subsetneq M_2\subsetneq \cdots$ an ascending chain of submodules that doesn't
		terminate. This means that for all $n\in\NN$ there is an element $f_n\in M_n-M_{n-1}$. Now
		consider the submodule $M'=\gen{f_1,f_2,f_3,\ldots}$. By hypothesis, $M'$ is finitely generated
		so that there is an $N\in\NN$ such that $M'=\gen{f_1,\ldots,f_N}$.

		However, since $f_n\in M_n$ for all $n$, this implies that
		\[
			M'=\gen{f_1,f_2,\ldots}=\gen{f_1,\ldots,f_N}\subseteq M_1\cup\cdots\cup M_N\subseteq M_N,
		\]
		but this is a contradiction because $f_{N+1}\not\in M_{N}$ by construction. Thus the ascending
		chain $M_1\subseteq M_2\subseteq \cdots$ must terminate.

	\item[$2\then 3$] Let $\Omega$ be a non-empty set of submodules of $M$ that does not have maximal
		elements with respect to inclusion. With this hypothesis we will construct an ascending chain
		that doesn't terminate and thus prove the implication $(2\then 3)$ by contradiction.

		Let $M_1\in\Omega$ be any submodule of $M$. Since $M_1$ is not maximal in $\Omega$, there is
		an $M_2\in\Omega$ such that $M_1\subsetneq M_2$. Likewise, since $M_2$ isn't maximal there is
		an $M_3\in\Omega$ such that $M_2\subsetneq M_3$. We can thus construct, inductively, an
		ascending chain $M_1\subsetneq M_2\subsetneq M_3\subsetneq \cdots$ that doesn't terminate.

	\item[$3\then 4$] Given a sequence $\{f_n\}_{n=1}^{\infty}$ of elements of $M$, define $M_n:=\gen{f_1,\ldots,f_n}$.
		By hypothesis, the set $\Omega:=\{M_1,M_2,\ldots\}$ has a maximal element, say $M_m$. However,
		for $n>m$ we have by construction $M_m\subseteq M_n$, but since $M_m$ is maximal, then $M_m=M_n$.
		In particular, $f_n\in M_m=\gen{f_1,\ldots,f_m}$ and thus $f_n$ is a linear combination of the
		set $\{f_1,\ldots,f_m\}$.

	\item[$4\then 1$] Let $M'$ be a submodule of $M$ that is not finitely generated. Then if $f_1\in M'$
		we have that $\gen{f_1}\neq M'$ because $M'$ is not finitely generated. Thus there exists an
		$f_2\in M'-\gen{f_1}$. Again, since $M'$ isn't finitely generated, we have $\gen{f_1,f_2}\subsetneq M'$.
		Inductively we can construct a sequence $f_1,f_2,f_3,\ldots$ such that
		\[
			\gen{f_1}\subsetneq \gen{f_1,f_2} \subsetneq \gen{f_1,f_2,f_3} \subsetneq \cdots.
		\]

		However, by hypothesis there is an $m\in\NN$ such that for all $n>m$, $f_n$ is a linear combination
		of $\{f_1,\ldots,f_m\}$ or in other words $f_n\in\gen{f_1,\ldots,f_m}$ for all $n>m$. This
		contradicts the construction of the sequence $f_1,f_2,\ldots$ becuase $f_{m+1}\not\in\gen{f_1,\ldots,f_m}$.
		Thus $M'$ must be finitely generated.
\end{enumerate}
%
\end{proof}%

