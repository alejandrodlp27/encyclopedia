\HRule
\subsection*{Exercise 9}
\subsection*{ %
Let $X$ be a set and $\Bb(X)$ be the power set of $X$ (the set of all subsets of $X$) together with the following
ring operations: for all $A,B\subseteq X$ define
\[
	A+B:= A\triangle B= (A-B)\cup(B-A) \quad\text{and}\quad A B:= A\cap B.
\]
We call $\Bb(X)$ the Boolean ring associated with the set $X$.
Prove the following:
\begin{enumerate}
	\item[$(i)$] $\Bb(X)$ is a commutative ring with unity whose zero element is $\emptyset$ and whose multiplicative identity
		is $X$.
	\item[(ii)] $\Bb(X)$ has only one unit, namely $X$.
	\item[(iii)] If $Y\subsetneq X$ is a proper set, then $\Bb(Y)$ is not a subring of $\Bb(X)$.
	\item[(iv)] If $I\subseteq\Bb(X)$ is a non-empty set, prove that:
		\[
			I \;\text{is an ideal} \quad\iff\quad \text{for any}\; A\in I \;\text{then} A'\in I \;\forall A'\subseteq A.
		\]
		In particular the principal ideal $\gen{A}$ is just the power set of $A$, ie.
		$\gen{A}=\{A'\in\Bb(X)\mid A'\subseteq A\}$.
	\item[(v)] Every maximal ideal $\m$ of $\Bb(X)$ is of the form $\gen{X-\{x\}}$ for some $x\in X$.
	\item[(vi)] Every prime ideal is maximal.
\end{enumerate}
} 

%%% ANSWER
\begin{proof} $\;$\\ %
(i) Clearly addition in $\Bb(X)$ is commutative and associative because the simetric difference of sets has these
properties:
\begin{align*}
	A\triangle B&=
	(A-B)\cup(B-A)=
	(B-A)\cup(A-B)=
	B\triangle A\\
	(A\triangle(B\triangle C)&=
	(A-(B\triangle C))\cup((B\triangle C)-A)=
	(A-((B-C)\cup(C-B)))\cup(((B-C)\cup(C-B))-A)\\&=
	\Big(A\cap\big((B\cap C^c)\cup(C\cap B^c)\big)^c\Big)\cup\Big(\big((B\cap C^c)\cup(C\cap B^c)\big)\cap A^c\Big)\\&=
	\Big(A\cap(B\cap C^c)^c\cap(C\cap B^c)^c\Big) \cup \Big( \big( (B^c \cup C)\cap(C^c\cup B)  \big)^c  \cap A^c\Big)\\&=
\end{align*}


%
\end{proof} %

